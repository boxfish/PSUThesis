%!TEX root = ../BoYu-Dissertation.tex
\graphicspath{{Figures/}}

\chapter{Conclusion} % (fold)
\label{cha:conclusion}
\section{Research contributions} % (fold)
\label{sec:contributions}
Focusing on the design of a computational system to support awareness in complex, distributed collaborative activities, this research fits into the design-science paradigm in information science \cite{Hevner2004}. As argued by Hevner et al \cite{Hevner2004}, effective design-science research must provide clear contributions in at least one of the following areas: (1) the development of constructs or models that extend and improve the understanding of the design problem (i.e. foundations), (2) the development of methods or tools that enable solutions to the design problem (i.e. design artifacts), (3) and the development or creative use of evaluation methods or new evaluation metrics (i.e. methodologies). In this study, we claim contributions in the first two areas. 

\paragraph*{The conceptual model of awareness in complex, distributed collaboration} % (fold)
\label{par:a_conceptual_model_of_awareness_in_complex_collaborations}
The first contribution of this research is the integrated conceptual model for understanding the awareness phenomena in complex, distributed collaboration. Our conceptual model is built on top of existing theories and models for understanding the awareness phenomena in the literature, and in turn contributes to existing knowledge foundations in two aspects:

\begin{enumerate}
	\item We adopt several interrelated constructs, i.e. activity, local scope, and dependency, to understand the \emph{product} of awareness phenomena, i.e. what part of the collaborative activity the actors should be aware of. These constructs together enrich the existing understanding of the awareness phenomena in collaborative environments. Beyond the knowledge sharing perspective, we emphasize the distributed nature of the awareness phenomena. Because of the differences in local scopes, each team member's awareness is partial, but at the same time is compatible for the team to perform collaborative activities successfully.
	\item We build on top of these constructs to understand the awareness \emph{process} in collaborative environments. Our model is able to account for how the awareness is distributed across multiple team members. Comparing with existing models, our model provides a better explanation of how the compatibility of different actors' awareness is achieved through the integration of multiple individual cognitive processes and social processes.
\end{enumerate}

Beyond the theoretical contribution, this conceptual model has also shown its value in guiding the design of awareness supporting tools in our study:

\begin{enumerate}
	\item First, it helps us to identify key design issues to support awareness in complex, distributed collaborative activities. By conceptualizing the awareness phenomena in distributed, complex collaboration as continuous developed through a variety of cognitive and social processes, the design issues for awareness support can be organized at both the individual level and the team level. The former focuses on supporting the cognitive processes of individual team members to develop their own awareness, while the latter provides support for the team processes in which team members interact with each other to achieve compatible awareness.
	\item Second, the conceptual model also guides our design of the computational awareness promotion approach. On one hand, the knowledge representation of our approach is designed to comply with the three constructs in the conceptual framework. As these constructs describe the configuration of the collaborative activity that regulates how the awareness is distributed across team members, they provide the necessary knowledge for the system to reason about human actors' awareness needs and hence promote awareness. Furthermore, one of the design principles of our awareness promotion is to consider the awareness support from a collective perspective and provide support for the developmental trajectories of collaborative awareness, which is motivated by the understanding of the awareness processes in the conceptual framework.
\end{enumerate}
% paragraph a_conceptual_framework_of_awareness_in_complex_collaborations (end)

\paragraph*{The computational awareness promotion approach} % (fold)
\label{par:the_awareness_promotion_approach}
The second major contribution of this study is the computational awareness promotion approach. Comparing with existing awareness support methods, the awareness promotion approach emphasizes the active role of the computer to mediate the awareness processes. While human actors still need to undergo the cognitive processes to develop individual awareness, and use it to make decision and perform actions in their own local scopes; the computer system takes the responsibility to maintain a collective picture of the whole collaborative activity, and utilizes this knowledge to facilitate the various cognitive and social awareness processes among human actors. 

Following these design principles, the awareness promotion approach is built on top of two major components: a computational representation of the field of work based on the PlanGraph model, and an event-driven model of the awareness processes. Then the computer system's behaviors to promote awareness are embedded in the interaction between these two components. On one hand is how the computer constructs and develops the knowledge representation of the field of work within the event-driven processes, and on the other hand is how the knowledge representation is used to promote these event-driven awareness processes.

The awareness promotion approach has several advantages to handle the increased level of complexity and dynamics in complex, distributed collaborative activities.

\begin{enumerate}
	\item First, the awareness promotion approach utilizes the computational knowledge representation to model the collaborative activities and offloads some of the representation and reasoning efforts from the human to the computer. Hence, it can  handle more complex collaborative configurations than existing awareness models.
	\item Meanwhile, the knowledge representation is dynamically updated to reflect the current state of the collaborative work, which allows it to handle increased level of dynamics.
	\item Last, as the computer's knowledge representation is at the team level and is equipped with the computational reasoning capabilities, it allows the computer to provide support on both the higher level of individual awareness processes and awareness transactions.
\end{enumerate}
% paragraph the_awareness_promotion_approach (end)
% section contributions (end)

\section{Comparison with existing studies} % (fold)
\label{sec:comparison_with_existing_studies}
Built on top of exiting theories and methods for awareness support, our approach share many commonalities with existing studies. However, it is important to distinguish our approach from several existing studies.

\subsection{Comparison with workflow systems} % (fold)
\label{sub:comparison_with_workflow_systems}
Our approach makes use of the computational knowledge representation of collaborative activities to promote awareness, which suggests that it is related to the concept of workflow systems \cite{Jansen2010,attie1996scheduling}. A workflow system uses a pre-determined model of business processes to guide and monitor progress through an activity. It can be used to model the goals, the decomposition of actions into sub-actions, dependencies among these actions, and actor roles and assigned responsibilities. For highly scripted business processes, a workflow model can be quite effective in decomposing and tracking an extended activity. However, workflow systems tend to break down in just the situations where activity awareness is most important — when opportunistic planning leads to creation or modification of goals or subgoals \cite{carroll2003a}. As the knowledge representation in our approach is mainly used to promote activity awareness, the model differs from the workflow model in two important ways:

\begin{enumerate}
	\item The purpose of our knowledge representation is to provide a form of resource for the system to record knowledge about the current collaborative activity, rather than a symbolic model to regulate what the actors should do in the activity.
	\item Following the SharedPlan theory, the construction of such a knowledge representation is a dynamic process. Unlike the workflow systems that use expert-derived models of existing business processes, the knowledge representation in our approach is dynamically developed as the actors interact with each other and with the system. In this way, the knowledge representation for each collaborative activity is unique and reflects the real situation in which the collaboration is embedded.
\end{enumerate}
% subsection comparison_with_workflow_systems (end)
\subsection{Comparing with space-based awareness models} % (fold)
\label{sub:comparing_with_space_based_awareness_models}
In Chapter \ref{sec:the_state_of_art}, we make the clear distinction between space-based and event-based awareness models. While the space-based model emphasizes the importance of shared representations for providing awareness information, the event-based model explicitly represents the awareness information as discrete events. Although we adopt the event-based approach in our awareness promotion framework to model  awareness processes, we do not neglect the importance of space-based models in supporting awareness. Actually, the review of existing awareness systems in Chapter \ref{sec:the_state_of_art} shows that most of awareness systems integrate both the space-based model and event-based model in the design. On one hand, the shared space provides the context to interpret awareness events. On the other hand, events provide the lightweight information unit to allow users to maintain the awareness in a `push' way, i.e. the users are only notified when there are some noteworthy events happening, so that they can focus on their individual work when nothing needs to be aware of.

In our event-driven awareness promotion framework, we also emphasize the visualization of shared representation to support awareness. However, beyond merely presenting the shared workspace that is visible to all team members, our approach makes use of the computational knowledge representation to provide shared representations at a more meaningful level:

\begin{enumerate}
	\item The activity view provides an overview of the activity structure. Such an external representation is very important for awareness development, as it provides an activity-related frame of reference for the users to interpret awareness events and evaluate impacts of these events.
	\item The event view tracks the historical and social development of each event, which allow the actors to understand how the awareness knowledge is developed in a period of time, and who else have contributed to its development.
\end{enumerate}
% subsection comparing_with_space_based_awareness_models (end)
\subsection{Comparing with event processing systems} % (fold)
\label{sub:comparing_with_event_processing_systems}
Our event-driven model of awareness processes shares some commonality with existing event processing systems, as both use the concept of event as the basic unit to organize and present awareness information. However, they also differ in several important ways:

\begin{enumerate}
   \item In existing event processing systems, the computer primarily plays the role to distribute events. It detects events from sensors in the environment or feedbacks of human actions, filters them out based on user subscriptions, and present them to the users. However, in our approach, the computer also consumes events. As to maintain the knowledge representation of collaborative activities, the computer needs to take events as input of its reasoning process, and use them to make inference and update its knowledge representation. 
   \item The concept of events in our approach has a much richer meaning than existing event processing systems. It is not only used to describe the occurrences in the environment and in human activities, but also the psychological experience of these occurrences as human actors perceive and interpret them. We make a clear distinction between real world occurrence, event, and awareness in Section \ref{ssub:the_concept_of_events}.
   \item Existing event-based models focus on the generation and presentation of events to the users, but our approach emphasizes the whole process of awareness development driven by events. This means we are not only concerned about how to select and present events to the users, but also how to support the interpretation of these events, and how new events are generated based upon existing ones.
   \item Existing event-based models treat events as discrete from each other. The system processes one individual event each time. After the event is disseminated to the corresponding actors, the processing on the event is done. However, as we conceptualize the awareness as undergoing continuous development, the events need to be considered as connected, and how they are built on top of each other needs to be tracked by the system.
\end{enumerate}
% subsection comparing_with_event_processing_systems (end)
% section comparison_with_existing_studies (end)
\section{Limitations} % (fold)
\label{sec:limitations_of_our_approach}
To constrain the research to a manageable level, this study is built on top of several assumptions, which may not always be the case . As a result, these assumptions impose limitations on our approach when it is applied to real collaborative situations.

\subsection{Disparity between work and benefit} % (fold)
\label{sub:disparity_between_work_and_benefit}
 Our human-computer collaboration approach is based on the assumption that all the team members are willing to contribute to the system, i.e. reporting new events, helping system interpret events, and modifying the computer's behaviors when necessary. However, such an assumption on the user's motivation can easily break down due to the disparity between work and benefit that is evidential in existing collaborative systems \cite{Grudin1994}. The system requires human users to do additional work to enter or process information so that the system can work properly. This kind of additional work is often performed by the user who cannot directly benefit from it. For instance, a user who reports a new event may finds it not directly relevant to his/her own work, rather can impact another user's work. Ideally, the awareness system is expected to provide a collective benefit, i.e. everyone can benefit from it. However, due to the different types of tasks, prior experience, roles, and assignments, some people may spend more effort doing this additional work, but receive less benefit from others. This will greatly undermine their motivation to contribute. 

 The provision of higher-level activity awareness elements can address this problem to some extent. By showing collaborators with overall team goals, dependencies between different actors' actions, the system can make the collective benefit visible so that the actors can develop a better understanding how each other's work can indirectly benefit each other. Even my additional work may not immediately benefit my own goals, it can help others to finish their work, who could in turn help me with my work. In addition, the interface of the system needs to be carefully designed so as to minimize the additional work that needs to be performed by the user.    
% subsection disparity_between_work_and_benefit (end)

\subsection{Difficulty of externalization } % (fold)
\label{sub:difficulty_of_externalization_}
Along with the assumption on user's motivation to contribute, our system is built on top of another assumption that the users should be capable of doing it. However, many aspects of awareness events in our system are intentional in the sense that the information or events that collaborators need to become aware of are the results of other actors' interpretation, i.e. they often reflect the state of someone else's mind. As argued in \cite{carroll2003a}, it is often much more difficult for people to explicitly externalize and broadcast their goals and plans, and even when they do so, it is not always useful to or welcomed by their collaborators.

One possible solution to tackle the difficulty of externalization, as suggested by Rittenbruch et al. \cite{Rittenbruch2007}, is to provide some predefined indicators that allow actors to externalize intentions by choosing the appropriate indicator. In this way, externalization can only involve a small number of interactions, like clicking a button or selecting a menu item. However, the set of intention indicators  need to account for a large variety of information that users need to express. They therefore need to be highly flexible and tailorable.
% subsection difficulty_of_externalization_ (end)

\subsection{Impact of communication structure} % (fold)
\label{sub:impact_of_communication_structure}
Our approach assumes that the communication structure among the collaborators should be flat. As argued by Powell \cite{powell2003neither}, in unstable or dynamic environments, flat, non-hierarchical structure is a more effective way of organizing because it allows workers to communicate based on the changing demands of the task. However, such an assumption becomes problematic in many organizations where the work structure is hierarchical \cite{hinds2006structures}. In flat communication structures, collaborators exchange their awareness knowledge merely based on the interdependencies between their tasks and overlaps of local scopes. However, when organizational hierarchy is introduced, it imposes constraints on how the awareness information flows among team members. They need to understand whether an awareness event should be reported, to whom the event should be propagated to, and how the event will impact people at different levels. 

Our activity model has the potential to accommodate the hierarchical communication structure as it provides the hierarchical view of collaborative activities where different actors can be associated with different levels of goals and actions. However, some critical social factors, such as roles, policies, and conventions, are not modeled in the system. In order to apply the system in more complex organizational structures,  we need to avoid the common assumption of a flat work environment and work with representative users whenever possible to develop sophisticated understandings of the social, political, and motivational factors within organizations \cite{Grudin1994}.
% subsection impact_of_communication_structure (end)
% section limitations_of_our_approach (end)

\section{Future directions} % (fold)
\label{sec:future_directions}
This study has taken a step towards addressing the challenges of supporting awareness in complex, distributed collaborative activities. However, due to the higher level of complexity and dynamics in these activities, awareness support becomes a much more difficult task and this study by no means can address all the issues. In the following of this section, we discuss the major limitations and future directions of this study. 

\paragraph*{Behavioral study to justify the conceptual model} % (fold)
\label{par:behavioral_study_to_understand_the_conceptual_model}
As this research follows the design-science paradigm \cite{Hevner2004}, the conceptual model of awareness as we presented in this study is mainly used to improve the understanding of the design problem and identify the design issues. Even though the model is built on top of existing theories and models for understanding the awareness phenomena in the literature, it has not been rigorously tested. Some of the design assumptions in the model, such as the coupling between awareness requirements and the activities, the construction of local scopes from intentions and capabilities, need to be justified in behavioral studies. As argued by Hevner et al. \cite{Hevner2004}, the design science and behavioral science studies should complement each other in information system research. On one hand, the behavioral study seeks to develop and justify theories and models that explain or predict human phenomena surrounding the design activities. On the other hand, the results of design science research provide new artifacts that enable the behavioral study to learn its practical implications. As a result, we believe that the computational approach in this study provides the platform to conduct future behavioral studies to understand the awareness phenomena in complex, distributed collaboration, which in turn will inform the future iterations of the design. 
% paragraph behavioral_study_to_understand_the_conceptual_model (end)

\paragraph*{Usability study of the computational approach} % (fold)
\label{par:usability_study_of_the_prototype_system}
Although we have applied several design evaluation methods in the case study to demonstrate the effectiveness of the awareness promotion approach. The evaluation is largely formative and focuses on the utility of the system. However, if we want to know how well the computational approach works in real situations, a usability study becomes relevant. Even though it is a challenging task to evaluate the usability of the whole system, we can design several smaller studies to test the usability of individual components, such as the usability of the visualization of event propagation view, or the direct manipulation interface for event externalization.
% paragraph usability_study_of_the_prototype_system (end)

\paragraph*{Extension to asynchronous collaboration} % (fold)
\label{par:extension_to_asynchronous_collaboration}
This study has been focused on supporting synchronous collaboration where the awareness events are notified to the actors whenever they occur. The awareness support becomes more interesting when the complex, collaboration is also asynchronous with a significant time span. In the asynchronous situations, it becomes even more important to maintain the developmental trajectories of the events as people are much easier to lose track of how the events are developed. Furthermore, in the collaborative activities with a longer developmental history, the higher level of the activity awareness, such as communities of practice, social capital, and human development, becomes more important and needs to be well supported \cite{carroll2006a}.
% paragraph extension_to_asynchronous_collaboration (end)

\paragraph*{Extension to multi-tasking} % (fold)
\label{par:extension_to_multi_tasking}
The knowledge representation based on PlanGraph model in this study focuses on modeling the collaborative activity with a single shared goal across team members. However, in real world complex collaboration, it is possible that one actor participates in multiple collaborative activities at the same time. The PlanGraph model can be extended to support the multi-tasking situations, where multiple collaborative activities can be modeled at the same time. The multi-tasking situations provide some interesting problems to the knowledge representation, such as how the local scopes of the work should be defined when multiple activities are active, how the dependencies may be posed on two actions of the same actor, but belong to two different collaborative activities.
% paragraph extension_to_multi_tasking (end)
% section future_work (end)
\section{Closing remarks} % (fold)
\label{sec:closing_remarks}
Awareness is an important aspect of supporting distributed collaborative activities, but extant research on computer supported awareness systems have focused mostly on relatively simple and well-defined collaborative environments. This study expands the horizon to supporting awareness in more complex, distributed collaborative settings. Following the design-research methodology, we developed an integrated conceptual model to understand the unique characteristics of awareness phenomena when the complexity of collaboration is scaled up. Built on top of the conceptual model, we proposed the awareness promotion framework that emphasizes the active role of computer to mediate awareness processes based on a formal knowledge representation of collaborative activities. The insights gained in this study further both our conceptual understanding of awareness phenomena in collaborative activities and our technical knowledge of how to design awareness tools that can support more complex collaborative activities.
% section closing_remarks (end)
% chapter conclusion (end)




 

