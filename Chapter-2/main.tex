%!TEX root = ../BoYu-Dissertation.tex
\graphicspath{{Figures/}}
\chapter{Understanding Awareness} % (fold)
\label{cha:understanding_awareness}
This chapter attempts to identify the major issues for understanding the awareness phenomena in large-scale distributed collaborative activities. A review of what is currently known about the awareness phenomena at both individual and team levels is presented, which provides the grounding work for our conceptual framework in next chapter.
\section{Overview} % (fold)
\label{sec:overview}
The concept of awareness has come to play a central role in CSCW research. However, what in CSCW labeled as `awareness' has little in common, besides the fact that it represents some aspect of human interaction that is important for successful collaboration \cite{schmidt2002a}. In a broad sense, two types of awareness can be distinguished in the CSCW research area: \emph{social} awareness and \emph{activity-oriented} awareness \cite{prinz1999a,schmidt2002a,carroll2003a}.

\emph{Social} awareness addresses the availability of different kinds of information about the social context of the team members, e.g. awareness about what they are doing, if they are talking to someone, if they can be disturbed etc. \emph{Social} awareness thus is conceived of as something that engenders ``informal serendipitous interactions'' \cite{hudson1996a} and ``a shared space for community building'' \cite{Dourish1992}. Awareness of the general social context is an important aspect of collaborative work, especially in domains where the the actors are engaged in cooperative work in a loose and broad sense, or domains where socialization is crucial \cite{schmidt2002a}.

However, when the tasks of collaborating actors become closely interdependent on each other, more urgent concerns need to be given to the aspect of \emph{activity-oriented} awareness. The \emph{activity-oriented} awareness focuses on practices through which actors seamlessly align and integrate their distributed and yet interdependent activities, e.g. awareness of things being done or in need of being done, of developments within the joint effort that may be advantageous or detrimental for one’s own work, of occurrence that makes one’s work more urgent or leads to changes to the intended course of actions, etc. \cite{schmidt2002a}. The major difference of \emph{activity-oriented} awareness from \emph{social} awareness is that it focuses on activities performed to achieve a specific shared goal \cite{carroll2003a} and the actor’s being interdependent in their work \cite{schmidt2002a}.

In this study, we focuses on the \textbf{activity-oriented} aspect of awareness in large-scale, distributed collaboration, which can be understood from two levels of analysis (Figure \ref{fig:two_levels_of_analysis}). 

\begin{figure}[htbp] %  figure placement: here, top, bottom, or page
   \centering
   \includegraphics{two_levels_of_analysis.pdf} 
   \caption{Two levels of analysis for understanding awareness}
   \label{fig:two_levels_of_analysis}
\end{figure}

\begin{enumerate}
   \item On one hand, because of the distributed nature of the collaborative activity, the awareness is associated with individual actors, as each actor must be able to perceive the elements in the environment or related to other actors' actions, and then reason their consequences on his/her individual work. The awareness at individual level provides the basis for the team members to understand each other's work, and thereafter to achieve awareness at the team level. Existing studies on awareness at the individual level focus on how the awareness of an actor is created through interaction between a person and his/her environment.
   \item On the other hand, there is a lot more to collaborative awareness than merely combining each team member's individual awareness \cite{salas1995situation}. To ensure that different actors' individual awareness is compatible with each other, and together results in a coordinated and complete team level awareness of the collaborative activity, the actors have to interact with each other through team processes. Existing studies in collaborative awareness have focused on explaining how the team level awareness is borne out of the interactions between individual actors. 
\end{enumerate}
The remainder of this chapter reviews existing studies related to these two levels of analysis respectively.
% section overview (end)


\section{Individual awareness}
\label{sec:awareness_in_individuals}
Research into awareness at the individual level originated from the study of situation awareness (SA) in the human factors research community. Situation awareness is considered as knowledge created through interaction between a person and his/her environment, i.e. ``knowing what is going on'' in the situation \cite{Endsley1995}. A good general definition of situation awareness is as ``the up-to-the minute cognizance required to operate or maintain a system'' \cite{Adams1995}. Although most of the situation awareness models in the literature are individual focused theories \cite{Salmon2008}, it has also been well recognized as an important element in collaborative environments. For example, Gutwin and Greeenberg \cite{Gutwin2002} view their workspace awareness as a specialization of situation awareness tied to the specific setting of the shared workspace. The concept of activity awareness proposed by Carroll et al. also subsumes situation awareness with an emphasis on aspects of the situation that have consequences for group work towards shared goals \cite{carroll2003a}. Hence, to understand the awareness phenomena in collaboration, it is important to start with understanding the practice of how individuals maintain and develop the situation awareness.

To understand the awareness at the individual level, there are two major aspects that are well recognized in the literature, and also applicable to collaborative environments: the \emph{product} of awareness \cite{Endsley1995} and the \emph{process} of gaining awareness \cite{Adams1995}. The former focuses on understanding what the awareness knowledge is composed of, and the latter is to understand how it is built and developed.

\subsection{The product of individual awareness} % (fold)
\label{sub:awareness_as_product}
Among the numerous attempts at specifying the product of situation awareness, i.e. what must be known to solve a class of problems posed when interacting with a dynamic environment \cite{Salmon2008}, Endsley's three-level model \cite{Endsley1995} has undoubtedly received the most attention. The three-level model describes situation awareness as the operator's internal model of the state of the environment, comprising three levels that is separate to the process used to achieve it \cite{Smith1995}:

\begin{enumerate}
	\item Level 1: \emph{perception of relevant elements in the environment}. An actor must first be able to gather perceptual information in the surrounding environment, and be able to selectively attend to those elements that are most relevant for the task at hand. At this stage, the information is merely perceived and no further processing takes place. 
	\item Level 2: \emph{Comprehension of task-related elements in Level 1}. Level 2 involves the interpretation of the perceptual information from Level 1 in a way that allows an actor to comprehend or understand its relevance in relation to their tasks and goals.
	\item Level 3: \emph{Projection of the states in the near future}. Using a combination of Level 1 and Level 2 awareness-related knowledge and experience in the form of mental models, actors forecast likely future states in the situation.
\end{enumerate}

Endsley's three-level model presents an intuitive description of situation awareness at the individual level and has been applied in a plethora of different domains \cite{Wickens2008}. Its simplicity and the division of SA into three levels allows the construct to be measured easily and effectively \cite{endsley1995measurement}, and also supports the abstraction of awareness requirements and the development of design guidelines \cite{Salmon2008}. Furthermore, it has been extended in order to describe team situation awareness \cite{endsley2001model}, and the three levels of awareness information are applicable in many collaborative situations \cite{Gutwin2002}.

Despite its popularity, the three-level mode has some important flaws. One of the key assumptions of the three-level model is the separation between depicting situation awareness as a product and the cognitive processes used to achieve it \cite{Salmon2008}, which leads to the inability to cope with the dynamic development of situation awareness \cite{Smith1995,uhlarik2002review}. Nevertheless, the model is also criticized by the ill-defined concept of mental models. Although Endsley's model emphasizes the critical roles of mental models in directing attention to critical elements in the environment (Level 1), integrating the elements to aid understanding of their meanings (Level 2), and generating possible future states (Level 3), the definition only includes the long-term knowledge that is formed by training and experiences, more important factors, such as the actor's goals, conceptual model of the current situation, are neglected \cite{Bedny1999}.
% subsection awareness_as_product (end)

\subsection{The development of individual awareness} % (fold)
\label{sub:awareness_as_process}
To address the dynamic development of situation awareness, many researchers have used Niesser's perceptual cycle model \cite{neisser1976cognition} to clarify the cognitive components involved in the acquisition and development of situation awareness \cite{Smith1995,Adams1995,Gutwin2002,Stanton2009}. According to the perceptual cycle model (Figure. \ref{fig:perceptual_cycle}), an actor's interaction with the world continues in an infinite cyclical nature. By perceiving the available information in the environment, the actor modifies its knowledge. Knowledge directs the actor's activity in the environment. That activity samples and perhaps anticipates or alters the environment, which in turn informs the actor. The informed, directed sampling and/or anticipation capture the essence of behavioral characteristic of situation awareness.

\begin{figure}[htbp] %  figure placement: here, top, bottom, or page
   \centering
   \includegraphics[width=5in]{perceptual_cycle.jpg} 
   \caption{Niesser's Perceptual Cycle Model \cite{Salmon2008}}
   \label{fig:perceptual_cycle}
\end{figure}

Based upon Niesser’s perceptual cycle model, Smith and Hancock suggest that situation awareness is neither resident in the world nor in the person, but resides through the interaction of the person with the world \cite{Smith1995}. Thus they viewing situation awareness as a generative process in `an adaptive cycle of knowledge, action and information' \cite{Smith1995}. In a similar fashion, Adams et al. \cite{Adams1995} used a modified version of Niesser’s perceptual cycle model to describe how situation awareness works. They argue that the process of achieving and maintaining situation awareness revolves around internally held mental models, which facilitate the anticipation of situational events, directing an actor's attention to cues in the environment and directing their eventual course of action. An actor then conducts checks to confirm that the evolving situation conforms to their expectations. Any unexpected events serve to prompt further search and explanation, which in turn modifies the actor's existing model. Gutwin and Greeenberg used the perception-action cycle to explain how the awareness is maintained in a shared workspace, in which awareness knowledge both directs and is updated by perceptual exploration of the workspace environment \cite{Gutwin2002}.

One of the key assumptions of these models based on the perceptual cycle is the interplay between the awareness information and the internal mental model of current situation. In the process of awareness development, some knowledge is activated and integrated into the mental model, while some becomes inactive or removed from the mental model. Although Smith and Hancock suggested that the adaptation of awareness information into the mental model should be goal-directed, i.e. it must reside in the task environment rather than in the actor's head \cite{Smith1995}, little detail has been given about the cognitive processes that guide the selection and interpretation of awareness information into the mental models.

In an attempt to clarify the cognitive processes involved in the development of indivisual awareness information, Bedny and Meister \cite{Bedny1999} propose a description of situation awareness based on the activity theory. They purport that individuals possess goals that represent an ideal image or desired end state of activity, which direct them towards the end state or methods of activity (or actions) that permit the achievement of these goals. It is the difference between the goals and the current situation that motivates an individual to engage in the awareness process and take action towards achieving the goal. They conceptualize activity in three stages: the orientational stage, the executive stage, and the evaluative stage. The orientational stage involves the development of an internal representation or picture of the world or current situation. The executive stage involves proceeding towards a desired goal via decision-making and action execution. Finally, the evaluative stage involves assessing the situation via information feedback, which in turn influences the executive and orientational components.

Based on the activity theory, Bedny and Meister emphasize how the actor's goals and activities play a central role in the process of awareness development. Critical environmental features are identified based upon their significance to the individual's activities and the individual’s motivation towards the task goal. The interpretation of these features modifies an individual’s goals and conceptual model of the current situation, which directs their activities and interaction with the world.
% subsection awareness_as_process (end)

\section{Collaborative awareness} % (fold)
\label{sec:awareness_in_collaboration}
The awareness phenomena in collaborative environments is indubitably more complex than awareness at the individual level. Beyond knowing what is going on in the environment, in their tasks, team members also need to develop an understanding of the activities of others, which provides a context of their own activities. This context is used to ensure that individual contributions are relevant to the group's shared goal as a whole \cite{dourish1992awareness}. The collaborative awareness is built on top of individual awareness, but at the same time requires the interaction between individuals to ensure that the combination of individual awareness together is sufficient for the team to perform the shared activity.

The awareness at the team level can also be analyzed from the `product' and the `process' aspects. From the `product' aspect, it is to understand the configuration of the collaborative awareness, i.e. how it is composed of the awareness of individual team members. From the `process' aspect, it is to understand the interaction between team members through which the collaborative awareness is developed. This section reviews three prominent conceptualizations of awareness at the collaborative level from both the `product' and the `process' aspects (Table \ref{tab:collaborative_awareness}).

{
   \footnotesize
   \begin{longtable}{>{\raggedright}p{1.1in}>{\raggedright}p{2.2in}>{\raggedright}p{2.2in}}
   \toprule 
    & \textbf{The `product' of awareness} & \textbf{The `process' of awareness}\tabularnewline
   \midrule 
   Collaborative awareness as shared knowledge & individual team member\textquoteright{}s awareness and the shared
   awareness knowledge with other team members & effective team processes for sharing relevant information, such as
   communication and mutual monitoring\tabularnewline
   \midrule 
   Collaborative awareness as shared activities & the shared context surrounding a collaborative activity, e.g. the shared
   goals and status, the top-down goal decomposition, the dependencies
   within the actions, etc. & the joint construction of common ground, shared practices, social
   capital, and human development\tabularnewline
   \midrule 
   Collaborative awareness as distributed cognition & distributed individual awareness that is overlapping, and complementary
   with each other & transactions representing the exchanging of awareness between actors\tabularnewline
   \bottomrule
   \caption{Conceptualizations of collaborative awareness}
   \label{tab:collaborative_awareness}

\end{longtable}

}

\subsection{Collaborative awareness as shared knowledge} % (fold)
\label{sub:team_situation_awareness}
The research on team situation awareness (TSA) attempts to extend the theories and models of situation awareness to collaborative settings. Most attempts to understand team SA have centered on a `shared understanding' of the same situation. Endsley et al. \cite{endsley2001model} suggest that, during team activities, situation awareness can overlap between team members, in that individuals need to perceive, comprehend and project awareness elements that are specifically related to their specific role in the team, but also elements that are required by themselves and by members of the team. Successful team performance therefore requires that individual team members have good situation awareness on their specific elements and also the same awareness for those elements that are shared. It is therefore argued that, at a simple level, the collaborative awareness comprises two separate but related components: individual team member's awareness, and the shared awareness with other team members (Figure \ref{fig:tsa}). 

\begin{figure}[htbp] %  figure placement: here, top, bottom, or page
   \centering
   \includegraphics[width=3.5in]{TSA.jpg} 
   \caption{Shared Situation Awareness (adapted from Endsley 1995 \cite{Endsley1995})}
   \label{fig:tsa}
\end{figure}

The team situation awareness adopts the knowledge-in-common view of shared mental models \cite{Mohammed2001}, i.e. it focuses on how the shared understanding of the same situation is developed. Nofi \cite{nofi2000defining}, for example, defines team SA as: `a shared awareness of a particular situation' and Perla et al. \cite{perla2000gaming} suggest that `when used in the sense of ``shared awareness of a situation'', shared SA implies that we all understand a given situation in the same way'. Shu and Furuta sugggested that TSA comprises both individual SA and mutual awareness and can be defined as `two or more individuals share the common environment, up-to-moment understanding of situation of the environment, and another person's interaction with the cooperative task' \cite{shu2005inference}. As a result, a critical factor of team situation awareness is to define the configuration of shared awareness requirements, i.e. to identify the shared knowledge that is needed by which team member, or by the whole team.

The heavy emphasis of collaborative awareness on shared knowledge is simplistic in large-scale distributed collaboration. As argued by Mohammed and Dumville, the knowledge-in-common view may be appropriate for only certain task domains and types of groups \cite{Mohammed2001}. In teams with high level of division of work, the distribution of knowledge and skills across the team typically is not uniform, as a result, a high level of overlapping knowledge in such teams might be inefficient. As we characterize the collaboration as highly distributed where collaborators play specialized roles or attain distinct knowledge in the course of joint activities, they experience the situation in different ways. So whilst some of the information required by two different team members may be `shared' in the sense that they both need to attend to it as part of their job, their resultant understanding and use of it is different \cite{Salmon2010}.

By considering the collaborative awareness as the shared knowledge among team members, the development of awareness at the team level relies on effective team processes for sharing relevant information. Salas et al. \cite{salas1995situation} propose a framework of team situation awareness that comprises two critical processes, individual situation awareness and team processes. The development of collaborative awareness, as a result, has a cyclical nature of developing individual SA, sharing SA with other team members and then modifying SA based on other team members’ SA. Most researchers have focused on communication as the key team process in the development of collaborative awareness. Nofi, for example, cites communication as the most critical element in the creation of shared SA \cite{nofi2000defining}. Salas et al. argue that team members acquire individual awareness, and then communicate this throughout the team, which leads to a common team understanding \cite{salas1995situation}. Entin and Entin note that communication is a prerequisite for high levels of team SA \cite{entin2000assessing}. Another key team process that is critical to team SA is the process of mutual monitoring, whereby team members monitor one another's activities, allowing the sharing of awareness information without explicit verbal communication \cite{Gutwin2002,Salmon2008}. Although it is recognized that an increased level of team processes, such as communication, mutual monitoring, will lead to enhanced levels of team situation awareness. However, the specific relationships between team situation awareness and team processes remains largely unexplained \cite{Salmon2008}.
% subsection team_situation_awareness (end)

\subsection{Collaborative awareness as shared activities} % (fold)
\label{sub:activity_awareness}
To address the problem of the `shared knowledge' view of collaborative awareness, Carroll et al. proposes a new framework for understanding awareness in collaborative environment, based on the concept of `activity awareness' \cite{carroll2003a,carroll2006a}. The major distinction between team situation awareness and activity awareness is that, in realistically complex circumstances, instead of merely sharing relatively static and stable constructs such as knowledge in common, people share their activities \cite{carroll2006a}. In framing activity awareness, they appropriate the concept of \emph{activity} from Activity Theory to emphasize that collaborators need to be aware of a whole, shared activity as complex, socially embedded endeavor, organized in dynamic hierarchies, and not merely aware of the synchronous and easily noticeable aspects of the activity \cite{Carroll2009}. 

Similar to the activity-directed SA model proposed by Bedny and Meister \cite{Bedny1999}, activity awareness emphasizes the importance of using the concept of activity to structure the products and processes of awareness phenomena. The ultimate motivation of human actors to acquire and maintain awareness in the collaborative environment is to achieve their shared goals by performing their activities. As a result, the context surrounding a collaborative activity, i.e. the manner in which a shared activity is decomposed into smaller inter-related tasks, how these subtasks are assigned or adopted by collaborators, and when and how distributed subtasks are interdependent on each other, becomes the most important aspects of the situation that the team members need to be aware of \cite{carroll2003a}.

By shifting the focus from shared knowledge to shared activities, activity awareness aligns the development of awareness with the development of collaborative activities. Most basically, activity awareness is achieved and developed through the joint construction of common ground - shared knowledge and beliefs, mutually identified and agreed upon by members through a rich variety of communication protocols \cite{carroll2006a}. In long-term, open-ended activities over significant spans of time, the construction of shared practices, social capital, and human development become also important to develop team member's activity awareness.

Activity awareness with its basis on Activity Theory, provides a level of abstraction that is more constructive and dynamic to structure the sharing requirement in collaborative awareness. The activity awareness focuses on the sharing of activities, i.e. the importance of a common picture of the shared collaborative activities. However, such a common picture is usually distributed in the whole group, instead of in any single actor's mind \cite{Stanton2009}. Each actor in the group has their own awareness, related to the goals they are working towards. However, this seldom includes the whole picture of the collaborative activity, and only when all the actors' awareness knowledge is meshed up together, the common picture emerges. Activity awareness framework provides little support to explain how the activity knowledge is distributed across multiple actors.
% subsection activity_awareness (end)

\subsection{Collaborative awareness as distributed cognition} % (fold)
\label{sub:distributed_team_awareness}
A more recent theme to conceptualize awareness in collaboration is the concept of distributed or systemic team awareness \cite{Stanton2009,artman1998situation}. Distributed team awareness approaches are borne out of distributed cognition theory \cite{hutchins1995cognition}, which describes the notion of joint cognitive systems comprising the people in the system and the artifacts that they use. Within such systems, cognition is achieved through coordination between the system units \cite{artman1998situation} and is therefore viewed as an emergent property (i.e. relationship between systemic elements) of the system rather than an individual endeavor. Distributed team awareness approaches therefore view awareness in collaboration not as a shared understanding of the situation, but rather as a characteristic of the socio-technical system itself \cite{artman1998situation}. Whilst recognizing that team members possess their individual SA for a particular situation and that they may share their understanding of the situation, distributed team awareness assume that awareness is distributed across the different human and technological agents involved in collaborative systems \cite{Stanton2009}.

The main difference between distributed team awareness and other TSA and activity awareness models relates to the concept of \emph{compatible} awareness \cite{Stanton2009}. Distributed team awareness postulates that, within collaborative systems, each team member does not need to know everything, rather they possess the awareness that they need for their specific tasks. Although different team members may be aware of the same information, this awareness is not shared, since the team members often have different goals and so view the situation differently based on their own tasks and goals. Different team member's individual awareness can be different in content but at the same time is compatible in that it is all collectively needed for the overall team to perform the collaborative activity successfully \cite{Salmon2010}. The team members' individual awareness can be overlapping, and complementary with each other, and hence deficiencies in one actor's individual awareness can be compensated by another actor. To use the analogy of a cog in a machine, each cog does not need to know about all the other cogs, rather it needs only to be able to connect with those cogs adjacent to it (Figure \ref{fig:compatible_awareness}). Thus it is suggested that `compatibility' is the key to collaborative awareness, rather than `sharedness' \cite{Salmon2008a}.

\begin{figure}[htbp] %  figure placement: here, top, bottom, or page
   \centering
   \includegraphics[width=3.1in]{compatible_awareness.jpg} 
   \caption{Compatible Awareness (adapted from Salmon et al. \cite{Stanton2009})}
   \label{fig:compatible_awareness}
\end{figure}

While the distributed team awareness emphasizes the distribution of awareness, it does not discount the interactions among different team members. The distributed team awareness is acquired and maintained through \emph{transactions} that arise from communications or other team processes \cite{Salmon2010}. A transaction in this case represents an exchange of awareness between actors. Actors receive information, integrate it with existing knowledge, and then pass on to other agents. The interpretation on that information changes per team member. The exchange of information between team members leads to transactions in the awareness being passed around. For example, an actor may perceive certain awareness element in the environment, interpret the meaning, and then pass it to another actor via a transaction. The second actor then builds its own interpretation upon the first actor's interpretation, and may start a new transaction to pass the awareness to other actors. Hence, it is the systemic transformation of awareness elements as they cross the local boundary from one team member to another that bestows upon awareness in collaboration an emergent behavior \cite{Stanton2009}.

The concept of distributed team awareness has been investigated in a number of domains, including naval warfare \cite{Stanton2006}, energy distribution \cite{Salmon2008a}, and air traffic control \cite{Stanton2009}. The major strength of the approach is related to the systemic approach that it advocates, which is more suitable to analyze the awareness phenomena in complex, real world collaborative activities \cite{Stanton2009}. However, the main weakness, as admitted by the authors, is also related to it complexity \cite{Salmon2010}. Similar to other team situation awareness models, it uses concepts as the basic unit to analyze awareness elements, which often leads to extremely large networks in order to represent all the concepts and their relationships. A possible remedy is to integrate the distributed team awareness with the activity-directed models and switch the basic unit of analysis from concepts to activities to understand the awareness phenomena.
% subsection distributed_team_awareness (end)
% section awareness_in_collaboration (end)

\section{Discussion} % (fold)
\label{sec:discussion}
By reviewing the existing theories and conceptualizations of the awareness phenomena, we believe that none of them alone can account for all the aspects of the awareness phenomena in large-scale distributed collaboration. 

\begin{enumerate}
   \item The models of situation awareness primarily focus on the individual level of the awareness phenomena, and have limited support for collaborative awareness. 
   \item We agree with the distributed team awareness approaches \cite{Salmon2010} on that, the conceptualization of collaborative awareness merely based on the idea of `sharing' is an oversimplification in large-scale distributed collaboration. As we characterize the collaboration as highly distributed where collaborators play specialized roles or attain distinct knowledge in the course of joint activities, the interaction between the actors to achieve collaborative awareness is more about transactions than information sharing.
   \item We resonate with the activity directed SA model \cite{Bedny1999} and the activity awareness framework \cite{carroll2003a} that existing models using concepts as the basic unit to analyze awareness elements tend to be too static and rigid for collaboration with higher level of dynamics.
\end{enumerate}

As a result, instead of adopting one particular viewpoint, we believe that a suitable conceptual model for understanding awareness in large-scale distributed collaborative activities should integrate multiple models in the literature. Specifically, we identify the following requirements for such a model of awareness:

\paragraph*{Integration of individual and collaborative awareness} % (fold)
\label{par:the_integration_of_individual_and_collaborative_awareness}
As most existing theories and models of awareness in collaboration claim that individual awareness is still an important component in collaborative environment, the conceptual model should be able to account for awareness at both the individual and the team levels, and emphasize on how these two aspects interplay with each other. On one hand, because of the distributed nature of the collaborative activity, awareness is associated with individual actors, as each actor must be able to achieve the individual awareness that provides the basis for them to understand each other’s work, and thereafter to achieve awareness at the team level. On the other hand, collaborative awareness requires the actors to interact with each other through team processes so that their individual awareness can be meshed up together.
% paragraph the_integration_of_individual_and_collaborative_awareness (end)

\paragraph*{Support for distributed awareness} % (fold)
\label{par:the_distributed_nature_of_awareness}
Because of the differences in goals, roles, the tasks being performed make, each team member’s awareness is different in content, but at the same time is compatible for the team to perform collaborative activities successfully. Hence, the conceptual model should be able to account for how the awareness is distributed across multiple team members, and meanwhile can interact with each other to achieve compatibility. 
% paragraph the_distributed_nature_of_awareness (end)

\paragraph*{Integration of awareness model and activity model} % (fold)
\label{par:the_coupling_between_awareness_and_activity}
The conceptual model should emphasizes the importance of using the concept of activity to structure the products and processes of awareness phenomena. As pointed out by Schmidt \cite{schmidt2002a}, when we are talking about `awareness', we are talking about the phenomena that actors align and integrate their actions with the actions of others to achieve their shared goals. As a result, any need for the actors to acquire and maintain awareness in the collaborative environment arises out of their need to perform their activities and not for its own sake.
% paragraph the_coupling_between_awareness_and_activity (end)

Next chapter shows our attempt to provide such an integrated conceptual model of awareness in large-scale distributed collaborative activities that can satisfy these requirements.
% section discussion (end)

% chapter understanding_coordination (end)
