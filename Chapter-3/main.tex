%!TEX root = ../BoYu-Dissertation.tex
\graphicspath{{Figures/}}

\chapter{Designing for Awareness Support} % (fold)
\label{cha:designing_for_awareness_support}
Following the conceptual model of awareness phenomena in complex collaborative activities, we describe the design space of awareness support in this chapter. The goal of this chapter is to identify key design issues to support awareness in distributed, complex collaborative activities, by mapping them to the major components in our conceptual model presented in Chapter \ref{cha:understanding_awareness}. We first present a taxonomy of the whole design space with the major design issues identified. Then we discuss how these design issues have been addressed (partially) in existing awareness support systems, and the design challenges that motivate our approach. 

\section{The Design Space} % (fold)
\label{sec:the_design_space}
By conceptualizing the awareness phenomena in distributed, complex collaboration as a continuous development interleaved with a variety of cognitive and social processes, the design issues for awareness support can be organized at two levels: the \emph{individual} level and the \emph{team} level. The former focuses on supporting the cognitive processes of individual team members to develop their own awareness, while the latter provides support for the social processes in which team members interact with each other to achieve awareness at the team level. Table \ref{tab:design_space} shows the taxonomy of the whole design space with the major design issues for each element identified.

{\footnotesize
	\begin{longtable}{>{\raggedright}m{1.2in}>{\raggedright}p{4in}}
\toprule 
Awareness Process & Design Issues\tabularnewline
\midrule 
Perception & \emph{1. Selection}: what awareness information should be made perceivable?

\emph{2. Presentation}: how should the awareness informaiton be made
perceivable to the users?

\emph{3. Delivery}: when should the awareness information be delivered
to the users?\tabularnewline
\midrule 
Comprehension & \emph{1. Representation}. How can the user's current goals and activities
be represented to enable comprehension? To what level of details?

\emph{2. Linking}. How can the system provide the linking between
awareness information and the current knowledge?

\emph{3. Externalization}. How can the system store the results of
comprehension? \tabularnewline
\midrule 
Projection & \emph{1. Representation}. How can the dependencies be represented
to enable projection? 

\emph{2. Reasoning}. How can the system provide the capabilities to
help users to perform the reasoning?

\emph{3. Externalization}. How can the system store the results of
projection? \tabularnewline
\midrule 
Feedthrough & 1. Coupling.

2. Level of Details.\tabularnewline
\midrule 
Manifestation & 1. Display. How can the system support the user to indicate what they
want to manifest?

2. Visibilit. How can the user to control who and what should be visible?\tabularnewline
\midrule 
Communication & 1. Recipient. Who should I communicate to?

2. Reference. What the user refers to?

3. Modality. \tabularnewline
\bottomrule
\caption{Summary of Design Space for Awareness Support}
\label{tab:design_space}
\end{longtable}}


\subsection{Designing for individuals} % (fold)
\label{sub:designing_for_individuals}
At the individual level, the role of computer support can be understood in term of designing `cognitive artifact' \cite{Norman1992}, which is defined as an artificial device designed to maintain, display, or operate upon information in order to aid cognition. Norman argues that an important design consideration for cognitive artifacts is the human action cycle that emphasizes the two sides of human action \cite{Norman1992}. One side is the `evaluation' side of action by perceiving, interpreting and evaluating the state of the environment. The other is the `execution' side that of acting upon the environment. Norman's human action cycle matches well with the major steps of awareness development at the individual level, where the `evaluation' side refers to the \emph{perception}, \emph{comprehension}, and \emph{projection}, and the `execution' side includes the \emph{decision making} and \emph{action execution}.

\subsubsection*{Perception} % (fold)
\label{ssub:perception}
The achievement of individual awareness starts with the ability of individuals to perceive key features in the environment. The primary goal of awareness support in the perception process is to ensure that the awareness information is perceivable to the users. This goal can be achieved from two aspects: \emph{selection} and \emph{presentation}. The former focuses on filtering out the information so that only the relevant information is perceived by the users, while the latter involves strengthening the stimulus to ensure that information is perceivable.

Support for \emph{selection} is based on the assumption that perception is a selective process that depends on the requirements of the current working situation and the tasks at hand \cite{Endsley1995}. Not all information is of relevance to a user, hence the pool of all awareness information has to be processed to filter the relevant information \cite{Berlage1999}. Support for selection is actually to delegate the effort of filtering out irrelevant information to computer systems, so that human actors can focus on processing only the relevant set of information to avoid information overload.

Support for \emph{presentation} is to strengthen the stimulus in the user interface to ensure important awareness information is perceivable. Existing studies in cognition have shown that the salience of elements in the environment will have a large impact on which portions of the environment are initially attended to, and these elements will form the basis for perception \cite{Hegarty2011}. As a result, the way in which information is presented via the interface will largely influence the perception process by determining which part of the environment will draw the user's attention.
% subsubsection perception (end)

\subsubsection*{Comprehension} % (fold)
\label{ssub:comprehension}
Comprehension is to establish the connection between the awareness information and the user's current goals and activities. \cite{oulasvirta2007a}

1. Representation. How can the user's current goals and activities be represented to enable comprehension? To what level of details?

2. Linking. How can the system provide the linking between awareness information and the current knowledge?

3. Externalization. How can the system store the results of comprehension?
% subsubsection comprehension (end)

\subsubsection*{Projection} % (fold)
\label{ssub:projection}
Projection is the process that the individual uses the dependencies to predict the future states of other activities based on the comprehension of awareness information.

1. Representation. How can the dependencies be represented to enable projection? 

2. Reasoning. How can the system provide the capabilities to help users to perform the reasoning?

3. Externalization. How can the system store the results of projection?

% subsubsection projection (end)
% subsection designing_for_individuals (end)

\subsection{Designing for the team} % (fold)
\label{sub:designing_for_the_team}

\subsubsection{Feedthrough} % (fold)
\label{ssub:feedthrough}

% subsubsection feedthrough (end)
\subsubsection{Communication} % (fold)
\label{ssub:communication}

% subsubsection communication (end)

\subsubsection{Manifestation} % (fold)
\label{ssub:manifestation}

% subsubsection manifestation (end)
% subsection designing_for_the_team (end)



% section the_design_space (end)

\section{The State of Art} % (fold)
\label{sec:the_state_of_art}

Systems to be reviewed:

DIVA \cite{springerlink:10.1023/A:1008608425504}

Event notification model (GroupDesk, PoLIAwaC et al.)
Spatial model
Focus/Nimbus Model
Diffusion
Atmosphere

Dimensions
Field of Work: activity, local scope, dependencies
Processes: perception, interpretation, decision-making, action, propagation
Interplay: 
whether the filed of work can be updated by the processes
whether the processes are supported by explicit representation of field of work

% section the_state_of_art (end)

\section{Design Challenges} % (fold)
\label{sec:design_challenges}
1. The Problem of Scaling Up

2. Handling Dynamics


% section design_challenges (end)

\section{Discussion and Summary} % (fold)
\label{sec:discussion_and_summary}
In the CSCW literature, there have been several attempts to identify the design space of awareness support in collaborative systems. The awareness checklist \cite{antunes2010a} has been proposed to identify the design elements based on the analysis of Quality Assurance. The design space is structured in a list of quality assurance categories, such as accessibility, mobility, virtuality etc. Unlike these existing approaches, in this chapter we structure the design space of awareness support following a goal-oriented approach. The design issues are identified based on the cognitive and social tasks that need to be performed in each step of the whole awareness development cycle. Such an approach allows us to decompose the overall goal of supporting awareness into smaller, more manageable components. On the other hand, the components comply with our integrated conceptualization of the whole awareness phenomena, which allow us to integrate them together to provide a systematic solution.

At the individual level, existing awareness systems have focused on supporting perception, the higher level of awareness processes are relatively less supported. A possible reason is that the higher-level awareness processes usually involve sophisticated cognitive and reasoning capabilities where the human can do better than the computer. As a result, most awareness systems leave the comprehension and projection to the users. However, in complex collaborative activities, as we discussed earlier, the scaling problem becomes significant, and hence it becomes much more important for the computer to provide functions to amplify and enhance human cognition, not only in the stage of perception, but also in comprehension and projection. 

At the team level, existing awareness systems have focused on either supporting the explicit communication among human collaborators, or different approaches to providing `shared spaces' representing the fields of work so that the actions of each other become visible to each other. However, less discussion has been given to support the awareness interaction via mutual displaying and monitoring, which actually is an even more important aspect of awareness in many complex, real world collaborative activities \cite{heath2002a}.

% section discussion_and_summary (end)

% chapter designing_for_awareness_support (end)




 

