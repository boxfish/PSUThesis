%!TEX root = ../BoYu-Dissertation.tex
\graphicspath{{Figures/}}

\chapter{Discussion} % (fold)
\label{cha:discussion}
In this chapter, we discuss the implications of this study from several perspectives: first we present the design philosophies behind our awareness promotion approach, then distinguish our approach from some related studies, and last we discuss the limitations of our approach.

\section{From awareness support to awareness promotion} % (fold)
\label{sec:from_awareness_support_to_awareness_promotion}
In order to address the unique challenges for supporting awareness in complex, distributed collaborative activities, in this study we propose the awareness promotion approach to emphasize the active role of computer systems to mediate awareness among collaborators. Our awareness promotion approach has its basis in two basic parent disciplines:  artificial intelligence (AI) and human-computer interaction (HCI). 
\begin{enumerate}
   \item From AI, we consider the computer as \emph{an active agent} \cite{Brown99activeuser}, situated within the collaborative environment that adaptively sense, acts, and reacts within this environment over time, to pursue its goal of promoting awareness.
   \item From HCI, we act on the premise that computers and humans have fundamentally asymmetric abilities \cite{Dalal1994}; therefore, we focus on developing divisions of responsibility that exploit the strengths and overcome the weaknesses of both, and utilizing interaction techniques to facilitate effective \emph{human-computer collaboration} \cite{Terveen1995}.
\end{enumerate}

\subsection{The role of computer: from tool to mediator} % (fold)
\label{sub:the_role_of_computer}
In existing awareness systems, the role of computer can be described in the metaphor of `tool'. The `tool' perspective emphasizes that the human achieves his/her goal through the computer application \cite{Bodker1997}. In term of supporting awareness, it means the human actors use the computer to achieve awareness. The tool perspective emphasizes the control on the human user's side, and how the computer is used to achieve awareness depends on the human user's own knowledge. For example, in space-based systems, the system relies on the user to monitor the shared space to perceive awareness information; in event-based systems, the user needs to explicitly express his/her interests so that the computer can filter out events. 

However, when the collaborative activities become more and more distributed and complicated, more is demanded of the user to control the computer. Each user usually only has partial knowledge about the whole collaborative situation, and it is unlikely that the user will have the complete knowledge to decide on what exactly should be done in order to achieve awareness as a group. The user may have no idea what background information should be attended to in order to interpret a piece of awareness information that is out of his/her local scope of work. Or he/she cannot decide on who should be notified when his/her activities are changed. As a result, the user either uses the partial knowledge to develop awareness that is often incomplete, vague, or even incorrect; or has to extend the mental model to represent more knowledge that out of his/her local scope of work, which inevitably increases the user's cognitive load. 

As a result, we believe that merely relying on the user to control the computer as a `tool' to achieve awareness in distributed, complex collaboration is ineffective. Alternatively, a better approach is to allow the computer to take some control away from the user and actively perform some tasks to help the user to achieve awareness. Instead of relying on the user's internal, partial knowledge to control the awareness development, the computer can maintain a much more complete knowledge representation of the whole collaborative activities at the systematic level, and make use of this knowledge to infer the user's need in various awareness processes so as to provide awareness support. In this way, the computer can be considered as a `mediator' actively engaged in the collaborative activities along with human actors, but with its own goal (i.e. to assist human actors to achieve awareness), its own mental model (i.e. the knowledge representation of the whole field of work), and its own behaviors (i.e. the capabilities to reason about the user's need in the various awareness processes and adapt interaction and information presentation techniques to support it).
% subsection the_role_of_computer (end)

\subsection{Human computer collaboration} % (fold)
\label{sub:human_computer_collaboration}
While considering the computer as an active actor to support awareness, it does not mean we can delegate all the tasks to the computer and build a completely automatic system. Existing studies in HCI and cognitive science have suggested that the relationship between human users and artificial intelligence should be treated as joint cognitive systems \cite{Dalal1994} or human-computer collaboration systems \cite{Terveen1995}, emphasizing the appropriate partition of responsibility between human actors and computer systems. This perspective of human computer interaction is based on two premises.

\begin{enumerate}
   \item It begins with the premise that computers and humans have fundamentally asymmetric abilities. The computer's strength lies in its ability to store more information than can be stored in the human's short-term memory, to perform faster data retrieval and processing, and to conduct more reliable low-level routine inferences \cite{Brown99activeuser}. On the other hand, the human's strength lies in the ability to integrate information from multiple sources to provide insight necessary to draw complex, higher level inferences, and the capability to handle unexpected situations with the help of long-term memory, experiences, and even intuition. As a result, the goal of computer support is actually to investigate the relationship between the cognitive characteristics of the human and the cognitive characteristics of the computer, and get the computer to complement the human \cite{Dalal1994}.
   \item It assumes that the human and the computer be cooperative to each other in the interaction. It means not only the computer will keep track of the human's activities and adapt its behaviors to support human performance, but also the human is willing to exposing his/her goals and activities to the computer, helping system maintain the knowledge representation, giving away control to the computer, and modifying the computer's behaviors when necessary \cite{Terveen1995}. The goal is not to shield users from the complexities of interaction, but rather to focus on the use of interaction techniques to facilitate effective human-computer collaboration.
\end{enumerate}

Our approach base on these two premises to propose supporting the awareness development in distributed, complex collaboration as a human computer collaborative system. While human actors still need to undergo the cognitive processes to develop individual awareness, and use it to make decision and perform activities in their own local scopes; the computer system take the responsibility to maintain a collective picture of the whole field of work, and utilize this knowledge to facilitate the various cognitive and social awareness processes among human actors.
% section from_awareness_support_to_awareness_promotion (end)

\section{Comparison with existing studies} % (fold)
\label{sec:comparison_with_existing_studies}
Built on top of exiting theories and methods for awareness support, our approach share many commonalities with existing studies. However, it is important to distinguish our approach from several existing studies.

\subsection{Comparison with workflow systems} % (fold)
\label{sub:comparison_with_workflow_systems}
Our approach makes use of the computational knowledge representation of collaborative activities to promote awareness, which suggests that it is related to the concept of workflow systems \cite{Jansen2010,attie1996scheduling}. A workflow system uses a pre-determined model of business processes to guide and monitor progress through an activity. It can be used to model the goals, the decomposition of actions into sub-actions, dependencies among these actions, and actor roles and assigned responsibilities. For highly scripted business processes, a workflow model can be quite effective in decomposing and tracking an extended activity. However, workflow systems tend to break down in just the situations where activity awareness is most important — when opportunistic planning leads to creation or modification of goals or subgoals \cite{carroll2003a}. As the knowledge representation in our approach is mainly used to promote activity awareness, the model differs from the workflow model in two important ways:

\begin{enumerate}
	\item The purpose of our knowledge representation is to provide a form of resource for the system to record knowledge about the current collaborative activity, rather than a symbolic model to regulate what the actors should do in the activity.
	\item Following the SharedPlan theory, the construction of such a knowledge representation is a dynamic process. Unlike the workflow systems that use expert-derived models of existing business processes, the knowledge representation in our approach is dynamically developed as the actors interact with each other and with the system. In this way, the knowledge representation for each collaborative activity is unique and reflects the real situation in which the collaboration is embedded.
\end{enumerate}
% subsection comparison_with_workflow_systems (end)
\subsection{Comparing with space-based awareness models} % (fold)
\label{sub:comparing_with_space_based_awareness_models}
In Chapter \ref{sec:the_state_of_art}, we make the clear distinction between space-based and event-based awareness models. While the space-based model emphasizes the importance of shared representations for providing awareness information, the event-based model explicitly represents the awareness information as discrete events. Although we adopt the event-based approach in our awareness promotion framework to model  awareness processes, we do not neglect the importance of space-based models in supporting awareness. Actually, the review of existing awareness systems in Chapter \ref{sec:the_state_of_art} shows that most of awareness systems integrate both the space-based model and event-based model in the design. On one hand, the shared space provides the context to interpret awareness events. On the other hand, events provide the lightweight information unit to allow users to maintain the awareness in a `push' way, i.e. the users are only notified when there are some noteworthy events happening, so that they can focus on their individual work when nothing needs to be aware of.

In our event-driven awareness promotion framework, we also emphasize the visualization of shared representation to support awareness. However, beyond merely presenting the shared workspace that is visible to all team members, our approach makes use of the computational knowledge representation to provide shared representations at a more meaningful level:

\begin{enumerate}
	\item The activity view provides an overview of the activity structure. Such an external representation is very important for awareness development, as it provides an activity-related frame of reference for the users to interpret awareness events and evaluate impacts of these events.
	\item The event view tracks the historical and social development of each event, which allow the actors to understand how the awareness knowledge is developed in a period of time, and who else have contributed to its development.
\end{enumerate}
% subsection comparing_with_space_based_awareness_models (end)
\subsection{Comparing with event processing systems} % (fold)
\label{sub:comparing_with_event_processing_systems}
Our event-driven model of awareness processes shares some commonality with existing event processing systems, as both use the concept of event as the basic unit to organize and present awareness information. However, they also differ in several important ways:

\begin{enumerate}
   \item In existing event processing systems, the computer primarily plays the role to distribute events. It detects events from sensors in the environment or feedbacks of human actions, filters them out based on user subscriptions, and present them to the users. However, in our approach, the computer also consumes events. As to maintain the knowledge representation of collaborative activities, the computer needs to take events as input of its reasoning process, and use them to make inference and update its knowledge representation. 
   \item The concept of events in our approach has a much richer meaning than existing event processing systems. It is not only used to describe the occurrences in the environment and in human activities, but also the psychological experience of these occurrences as human actors perceive and interpret them. We make a clear distinction between real world occurrence, event, and awareness in Section \ref{ssub:the_concept_of_events}.
   \item Existing event-based models focus on the generation and presentation of events to the users, but our approach emphasizes the whole process of awareness development driven by events. This means we are not only concerned about how to select and present events to the users, but also how to support the interpretation of these events, and how new events are generated based upon existing ones.
   \item Existing event-based models treat events as discrete from each other. The system processes one individual event each time. After the event is disseminated to the corresponding actors, the processing on the event is done. However, as we conceptualize the awareness as undergoing continuous development, the events need to be considered as connected, and how they are built on top of each other needs to be tracked by the system.
\end{enumerate}
% subsection comparing_with_event_processing_systems (end)
% section comparison_with_existing_studies (end)
\section{Limitations of our approach} % (fold)
\label{sec:limitations_of_our_approach}
To constrain the research to a manageable level, this study is built on top of several assumptions, which may not always be the case . As a result, these assumptions impose limitations on our approach when it is applied to real collaborative situations.

\subsection{Disparity between work and benefit} % (fold)
\label{sub:disparity_between_work_and_benefit}
 Our human-computer collaboration approach is based on the assumption that all the team members are willing to contribute to the system, i.e. reporting new events, helping system interpret events, and modifying the computer's behaviors when necessary. However, such an assumption on the user's motivation can easily break down due to the disparity between work and benefit that is evidential in existing collaborative systems \cite{Grudin1994}. The system requires human users to do additional work to enter or process information so that the system can work properly. This kind of additional work is often performed by the user who cannot directly benefit from it. For instance, a user who reports a new event may finds it not directly relevant to his/her own work, rather can impact another user's work. Ideally, the awareness system is expected to provide a collective benefit, i.e. everyone can benefit from it. However, due to the different types of tasks, prior experience, roles, and assignments, some people may spend more effort doing this additional work, but receive less benefit from others. This will greatly undermine their motivation to contribute. 

 The provision of higher-level activity awareness elements can address this problem to some extent. By showing collaborators with overall team goals, dependencies between different actors' actions, the system can make the collective benefit visible so that the actors can develop a better understanding how each other's work can indirectly benefit each other. Even my additional work may not immediately benefit my own goals, it can help others to finish their work, who could in turn help me with my work. In addition, the interface of the system needs to be carefully designed so as to minimize the additional work that needs to be performed by the user.    
% subsection disparity_between_work_and_benefit (end)

\subsection{Difficulty of externalization } % (fold)
\label{sub:difficulty_of_externalization_}
Along with the assumption on user's motivation to contribute, our system is built on top of another assumption that the users should be capable of doing it. However, many aspects of awareness events in our system are intentional in the sense that the information or events that collaborators need to become aware of are the results of other actors' interpretation, i.e. they often reflect the state of someone else's mind. As argued in \cite{carroll2003a}, it is often much more difficult for people to explicitly externalize and broadcast their goals and plans, and even when they do so, it is not always useful to or welcomed by their collaborators.

One possible solution to tackle the difficulty of externalization, as suggested by Rittenbruch et al. \cite{Rittenbruch2007}, is to provide some predefined indicators that allow actors to externalize intentions by choosing the appropriate indicator. In this way, externalization can only involve a small number of interactions, like clicking a button or selecting a menu item. However, the set of intention indicators  need to account for a large variety of information that users need to express. They therefore need to be highly flexible and tailorable.
% subsection difficulty_of_externalization_ (end)

\subsection{Impact of communication structure} % (fold)
\label{sub:impact_of_communication_structure}
Our approach assumes that the communication structure among the collaborators should be flat. As argued by Powell \cite{powell2003neither}, in unstable or dynamic environments, flat, non-hierarchical structure is a more effective way of organizing because it allows workers to communicate based on the changing demands of the task. However, such an assumption becomes problematic in many organizations where the work structure is hierarchical \cite{hinds2006structures}. In flat communication structures, collaborators exchange their awareness knowledge merely based on the interdependencies between their tasks and overlaps of local scopes. However, when organizational hierarchy is introduced, it imposes constraints on how the awareness information flows among team members. They need to understand whether an awareness event should be reported, to whom the event should be propagated to, and how the event will impact people at different levels. 

Our activity model has the potential to accommodate the hierarchical communication structure as it provides the hierarchical view of collaborative activities where different actors can be associated with different levels of goals and actions. However, some critical social factors, such as roles, policies, and conventions, are not modeled in the system. In order to apply the system in more complex organizational structures,  we need to avoid the common assumption of a flat work environment and work with representative users whenever possible to develop sophisticated understandings of the social, political, and motivational factors within organizations \cite{Grudin1994}.
% subsection impact_of_communication_structure (end)
% section limitations_of_our_approach (end)
% chapter conclusion (end)




 

