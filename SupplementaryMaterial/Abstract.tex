% Place abstract below.
Maintaining awareness is central to effective coordination in distributed collaborative activities. Existing awareness solutions work for relatively small-scale collaboration in traditional workspace, and they suffer from either inflexibility or lack of scalability if applied to large-scale distributed collaborative activities that are characterized by higher level of complexity and dynamics, such as emergency response or medical systems. The overall objective of this study is to address the major challenges of awareness support in these large-scale and highly distributed collaborative activities.

In order to achieve the research objective, this study follows the design science paradigm, i.e. understanding of awareness problems in large-scale collaboration and their solutions are achieved through a set of design activities to develop useful and usable awareness supporting systems. Based on the grounding work in literature, we develop an integrated conceptual model to understand the awareness phenomena in large-scale distributed collaboration. This model helps us to generate concrete design issues that need to be addressed, identify knowledge gaps in existing studies, and guide the design of the computational awareness promotion framework. Following the computational framework, we develop the prototype system, EDAP, and perform a case study in emergency response to validate the approach and demonstrate its feasibility and utility.

This study has made two major contributions towards the research objective. Firstly, it provides an integrated conceptual model on top of several interrelated constructs, i.e. activity, local scope, and dependency, to understand the distributed nature of the awareness phenomena. The model is able to account for how the awareness is distributed across multiple team members through the integration of individual cognitive processes and team processes. Secondly, it presents a computational framework for awareness promotion. Comparing with existing awareness supporting systems, the awareness promotion approach emphasizes the active role of the computer to mediate awareness processes based on a formal knowledge representation of collaborative activities. The awareness promotion approach has shown several advantages to handle the increased level of complexity and dynamics in large-scale distributed collaborative activities: (1) it utilizes the computational knowledge representation to model collaborative activities and offloads some of the representation and reasoning efforts from the human to the computer; (2) the knowledge representation is dynamically updated to reflect current state of the collaborative work, which allows it to handle a variety of dynamics; (3) as it represents the collective knowledge about collaborative activities at the team level, it can provide support to mediate awareness transactions among multiple team members.