% Place abstract below.
One of the major challenges to support complex, distributed geo-collaborative activities is to integrate effective coordination mechanisms to manage different types of work dependencies. This study focuses on event-based awareness mechanisms to support the management of dependencies by giving team members an awareness of what each other are doing or have done so that participants can adjust and coordinate their work in a more flexible way. Unlike existing event-based mechanisms, this study (1) relies on a deep understanding of the variety and dynamics of dependencies in these activities to distribute awareness events, and (2) provides explicit visualization and interaction support for the interpretation of awareness events.  

The central idea of the approach to modeling dependencies and awareness events is based on a formal structure of collaborative activities. Rooted in the SharedPlans theory \cite{grosz1996collaborative}, the activity model treats a collaborative activity as an evolving shared plan situated in a set of physical and mental contextual factors. Such a model of collaborative activities then can be used to (1) interpret awareness events according to how they are related to the current activities, and notify the relevant users of the events based on the identification of dependencies between the events and users' current focuses. 

Then this study illustrates the visualization and interaction provided by the activity model through a design scenario, where a couple of first responders in an emergency response team can use it to help generate and interpret awareness information to manage dependencies. To investigate the impacts of complexities in geo-collaborative activities on the actors' ability to interpret awareness events, an experiment is being designed and performed. The general hypothesis is that the effectiveness of our visualization and interaction design to support awareness interpretation is correlated to the level of complexity of collaborative activities.
