% Place abstract below.
Maintaining awareness is central to effective coordination in distributed collaborative activities. Existing awareness solutions work for relatively small-scale collaboration in traditional workspace, and they suffer from either inflexibility or lack of scalability if applied to large-scale distributed collaborative activities that are characterized by higher level of complexity and dynamics, such as emergency response or medical systems. The overall objective of this study is to addresses the major challenges of awareness support in these large-scale and highly distributed collaborative activities.

To achieve the research objective, this study follows the design science paradigm. Knowledge and understanding of the aforementioned awareness problems in large-scale collaboration and their solutions are achieved through a set of design activities to develop useful and usable awareness systems. Based on the grounding work in the literature, we develop an integrated conceptual framework to understand the awareness phenomena in large-scale distributed collaboration. Such a conceptual framework help us to develop concrete design issues that need to be addressed, identify knowledge gaps in existing studies, and guide our design of the awareness promotion approach. Following the computational approach, we develop the prototype system to prove the feasibility of the approach. Then we perform the case study in a concrete emergency response scenario to demonstrate the utility of our approach in support awareness in large-scale distributed collaboration.

This study has made two major contributions towards the research objective. Firstly, this study provides an integrated conceptual model for understand the awareness phenomena in complex collaborations. Our conceptual model adopts several interrelated constructs, i.e. activity, local scope, and dependency, to understand the distributed nature of the awareness phenomena and is able to account for how the awareness is distributed across multiple team members through the integration of individual cognitive processes and team processes. Secondly, this study presents the computational framework for awareness promotion. Comparing with existing awareness support methods, the awareness promotion approach emphasizes the active role of the computer to mediate the awareness processes based on a formal representation of the collaborative activities. The awareness promotion approach has several advantages to handle the increased level of complexity and dynamics in large-scale distributed collaborative activities: (1) it utilizes the computational knowledge representation to model the collaborative activities and offloads some of the representation and reasoning efforts from the human to the computer; (2) the knowledge representation is dynamically updated to reflect the current state of the collaborative work, which allows it to handle increased level of dynamics; (3) as it represents the activity knowledge at the team level and is equipped with reasoning capabilities, it can provide support for both the higher level of individual awareness processes and awareness transactions at the team level.
