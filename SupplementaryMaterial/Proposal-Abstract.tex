% Place abstract below.
Cartographic representations of geographic information (i.e. maps) have been used in many collaborative decision-making activities (e.g. crisis management, urban planning, and environment protection). However, current Geographic Information Systems (GIS) designed for individual use have limitations to support the generation of cartographic representations in geo-collaborative activities. First, cartographic representations are used, not only to visualize geographic objects, but also to share and compare different perspectives, and negotiate solutions. Second, cartographic representations should be situated in ongoing collaborative activities. As an activity proceeds, the information that is relevant to the group and the appropriate visual forms to represent the information keep changing. 

To address these new challenges, this study focuses on the elaboration of an adaptive design methodology of cartographic visualization tools to support geo-collaboration, by integrating relevant theories and methods in CSCW, AI, cartography and GIS. Visualization systems following this approach should not only generate cartographic representations that support geo-collaboration, but also be highly adaptive, which means that they can provide a degree of intelligence in choosing proper visualization strategies dynamically as the collaborative activities progress. To achieve this goal, two major issues will be addressed: 1) an activity-oriented model of geo-collaborative context and a collaborative plan-based approach to represent and keep track of context will be discussed. The context model allows visualization systems to actively maintain the awareness of ongoing collaborative activities and generate visual representations accordingly. 2) This study considers the visualization process as part of the geo-collaboration and proposes a PlanGraph-based approach that establishes the adaptation methods for visualization of geographic information and makes cartographic decisions based on the collaborative context model. 

Distinguished from current approaches, this study emphasizes the active role of cartographic visualization in supporting geo-collaboration. The proposed approach will be implemented in a spoken dialogue-based prototype system, which will be used in experiments to evaluate its ability to reduce interaction complexity and improve user performance. The proposed research will make contributions to the research in both geo-collaboration and cartography communities.
